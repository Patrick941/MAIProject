\documentclass[12pt]{extarticle}
\usepackage{graphicx}
\usepackage{setspace}
\usepackage{xcolor}
\usepackage[a4paper, total={6in, 8in}]{geometry}
\usepackage{pgfgantt}

\begin{document}

\doublespacing

\title{\textbf{Debugging Problems}}
\author{Patrick Farmer\\ Supervisor: Dr. Jonathan Dukes}
\date{\today}
\maketitle

\begin{figure}[h!]
\centering
\includegraphics[width=\linewidth]{Images/Trinity_Main_Logo.jpg}
\label{fig:logo}
\end{figure}

\begin{center}
\large A Dissertation submitted in partial fulfillment of the requirements for the degree of MAI in Computer Engineering.
\end{center}

\newpage
\title {\Huge \textbf{\textcolor{blue}{Declaration}}}

\vspace{0.5cm}
\small I hereby declare that this Dissertation is entirely my own work and that it has not been submitted as an exercise for a degree at this or any other university.

\vspace{0.5cm}
\small I have read and I understand the plagiarism provisions in the General Regulations of the University Calendar for the current year, found at http://www.tcd.ie/calendar.
\vspace{0.5cm}

\small I have completed the Online Tutorial on avoiding plagiarism `Ready Steady Write', located at http://tcd-ie.libguides.com/plagiarism/ready-steady-write.
\vspace{0.5cm}

\small I consent to the examiner retaining a copy of the thesis beyond the examining period, should they so wish (EU GDPR May 2018).
\vspace{0.5cm}

\small I agree that this Dissertation will not be publicly available, but will be available to TCD staff and students in the University’s open access institutional repository on the Trinity domain only, subject to Irish Copyright Legislation and Trinity College Library conditions of use and acknowledgement.
\vspace{2cm}

\small Signed:~Patrick Farmer\hfill Date:~\today

\newpage
\tableofcontents

\newpage
\section{Acronyms}

\newpage
\section{Introduction}

This is a brief description of the project. It will also have a guide to the repository

\subsection{Context}
\subsection{Goals}
\subsection{Outcome}
\subsection{Evaluation Metrics}
\subsection{Structure of the Dissertation}

\section{Background}

This will dicuss the need for the project and similar projects that have been done in the past

\subsection{Literature Review}
Literature review of both debugging, teaching debugging and LLMs
\subsection{Summary of LLMs}


\section{Design and Implementation}

This will discuss the design and implementation of the project. This will include a diagram at the beginning showing the heirarchy of the project.

\subsection{Design Overview}
\subsection{Ollama}
\subsection{Code generation}
\subsection{Test case generation}
\subsection{Bug Insertion}
\subsubsection{LLM Bug Insertion}
\subsubsection{AST Bug Insertion}

\section{Testing and Evaluation}

This will discus the testing of the project which will show the improvement of the project over time. This will also include an evalution of the final state of the project.

\subsection{Metrics}
\subsubsection{Code Complexity}
\subsubsection{Code Diversity}
\subsubsection{Attempt Count}
\subsubsection{Run Time}

\section{Conclusion}

This will briefly summarise the project and discuss the future of the project.

\section{References}
\begin{thebibliography}{9}

\bibitem{jadud2006}
Jadud, M. C. (2006). Methods and tools for exploring novice compilation behaviour. Proceedings of the Second International Workshop on Computing Education Research, 73–84. https://doi.org/10.1145/1151588.1151600

\bibitem{li2019}
Li, C., Chan, E., Denny, P., Luxton-Reilly, A., \& Tempero, E. (2019). Towards a Framework for Teaching Debugging. Proceedings of the Twenty-First Australasian Computing Education Conference, 79–86. https://doi.org/10.1145/3286960.3286970

\bibitem{odell2017}
O’Dell, D. H. (2017). The Debugging Mindset: Understanding the psychology of learning strategies leads to effective problem-solving skills. Queue, 15(1), 71–90. https://doi.org/10.1145/3055301.3068754

\bibitem{parkinson2024}
Parkinson, M. M., Hermans, S., Gijbels, D., \& Dinsmore, D. L. (2024). Exploring debugging processes and regulation strategies during collaborative coding tasks among elementary and secondary students. Computer Science Education, 0(0), 1–28. https://doi.org/10.1080/08993408.2024.2305026

\bibitem{whalley2021}
Whalley, J., Settle, A., \& Luxton-Reilly, A. (2021). Analysis of a Process for Introductory Debugging. Proceedings of the 23rd Australasian Computing Education Conference, 11–20. https://doi.org/10.1145/3441636.3442300

\bibitem{whalley2023}
Whalley, J., Settle, A., \& Luxton-Reilly, A. (2023). A Think-Aloud Study of Novice Debugging. ACM Transactions on Computing Education, 23(2), 1–38. https://doi.org/10.1145/3589004

\bibitem{denny2023}
Denny, P., Leinonen, J., Prather, J., Luxton-Reilly, A., Amarouche, T., Becker, B. A., \& Reeves, B. N. (2023). Promptly: Using Prompt Problems to Teach Learners How to Effectively Utilize AI Code Generators. https://doi.org/10.48550/ARXIV.2307.16364

\bibitem{denny2024}
Denny, P., Leinonen, J., Prather, J., Luxton-Reilly, A., Amarouche, T., Becker, B. A., \& Reeves, B. N. (2024). Prompt Problems: A New Programming Exercise for the Generative AI Era. Proceedings of the 55th ACM Technical Symposium on Computer Science Education V. 1, 296–302. https://doi.org/10.1145/3626252.3630909

\bibitem{nguyen2024}
Nguyen, S., Babe, H. M., Zi, Y., Guha, A., Anderson, C. J., \& Feldman, M. Q. (2024). How Beginning Programmers and Code LLMs (Mis)read Each Other. Proceedings of the CHI Conference on Human Factors in Computing Systems, 1–26. https://doi.org/10.1145/3613904.3642706

\end{thebibliography}

\end{document}