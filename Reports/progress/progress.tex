\documentclass[12pt]{extarticle}
\usepackage{graphicx}
\usepackage{setspace}
\usepackage{xcolor}
\usepackage[a4paper, total={6in, 8in}]{geometry}

\begin{document}

\doublespacing

\title{\textbf{Debugging Problems}}
\author{Patrick Farmer\\ Supervisor: Dr. Jonathan Dukes}
\date{\today}
\maketitle

\begin{figure}[h!]
\centering
\includegraphics[width=\linewidth]{Images/Trinity_Main_Logo.jpg}
\label{fig:logo}
\end{figure}

\begin{center}
\large A Progress Report submitted in partial fulfillment of the requirements for the degree of MAI in Computer Engineering.
\end{center}

\newpage
\title {\Huge \textbf{\textcolor{blue}{Declaration}}}

\vspace{0.5cm}
\small I hereby declare that this Progress Report is entirely my own work and that it has not been submitted as an exercise for a degree at this or any other university.

\vspace{0.5cm}
\small I have read and I understand the plagiarism provisions in the General Regulations of the University Calendar for the current year, found at http://www.tcd.ie/calendar.
\vspace{0.5cm}

\small I have completed the Online Tutorial on avoiding plagiarism `Ready Steady Write', located at http://tcd-ie.libguides.com/plagiarism/ready-steady-write.
\vspace{0.5cm}

\small I consent / do not consent to the examiner retaining a copy of the thesis beyond the examining period, should they so wish (EU GDPR May 2018).
\vspace{0.5cm}

\small I agree that this thesis will not be publicly available, but will be available to TCD staff and students in the University’s open access institutional repository on the Trinity domain only, subject to Irish Copyright Legislation and Trinity College Library conditions of use and acknowledgement.  \textbf{Please consult with your supervisor on this last item before agreeing, and delete if you do not consent}
\vspace{2cm}

\small Signed:~Patrick Farmer\hfill Date:~\today

\newpage
\tableofcontents

\newpage
\section{Introduction}

Due to the developments in LLM capabilities over the past few years have introduced a couple of issues with the classic approach for teaching program. The first of which is the issue of plagiarism from students using AI generated code. Tasks that are often given to novice programmers can be completed within moments by chat GPT so a lot of students will choose this easier alternative.\\
There is then on the other hand the issue that AI assisted programming will likely exist in every job in the coming years, meaning that being able to work effectively with AI is also a very valuable skill. This project aims to address both of these issues by creating a tool that can generate defective code which the student must then debug.\\
This solves the issue of plagiarism as AI will struggle a lot more to fix broken code than it will to generate it.\\
It also teaches the student how to debug which is the most important tool when working with AI generated code.

\newpage
\section{Goals and Objectives}

You should clearly state the overall goal of your project (recall SMART goals from  EE4E03 Research Methods). You should also give your project objectives. These should be more specific actions that you expect to do (e.g. simulate, test, compare). At the end of the project it should be possible to assess which objectives have been met. In your final dissertation, you will have to assess your achievements referring back to these objectives – so think carefully about them. [0.5-1 page] 

\newpage
\section{Literature Review}

The literature review should overview relevant work in the domain of your project and provide motivation for the proposed work. A good literature review will demonstrate academic input through critical appraisal and a logical grouping of, and demonstration of links between, the literature. Make it clear how your proposed project sits in the space you have described, or how you are addressing shortcomings in other approaches etc. Reference all papers appropriately. [2-4 pages] 

\newpage
\section{Current state of the project}

To include preliminary simulations, experiments etc. This section should not turn into a detailed report on the experiments – that’s for the final dissertation. Highlight some key experiments you are designing or running, or findings to date. [1-2 pages]   

\newpage
\section{Project Management}

This should include a programme of work to be completed for the rest of the project. Units of time will most likely be days for the more immediate tasks, but more likely to be weeks (e.g. 1 week, 0.5 weeks) further out on your horizon. This will ideally link clearly to the original objectives of the project, but will provide more detail on what work will be done. Include key milestones, dependencies where appropriate, with dates. Include the dates of your presentation and dissertation submission in your plan (see module descriptor). The programme of work should also be supported by a detailed Gantt chart. You must include a proper risk assessment where you identify the major risks to your project and identify adequate contingencies. Again, your EE44E03 lectures will help you here. [2-3 pages] 

\newpage
\section{Conclusion}

\newpage
\section{Acknowledgements}

\end{document}